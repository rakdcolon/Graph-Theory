\documentclass[12pt,reqno]{amsart}

\usepackage{graphicx}

\usepackage{amssymb}
\usepackage{amsthm}
\theoremstyle{plain}

\newtheorem*{definition}{Definition}
\newtheorem*{axiom}{Axiom}
\newtheorem*{theorem}{Theorem}
\newtheorem*{corollary}{Corollary}
\newtheorem*{lemma}{Lemma}
\newtheorem*{example}{Example}
\newtheorem*{proposition}{Proposition}
\usepackage{lineno}

\title{Honors Real Analysis Lecture 21}
\author{Rohan Karamel}

\begin{document}

    \begin{abstract}
        This lecture reviews material to prepare for the upcoming second midterm.
    \end{abstract}

    \maketitle

    \begin{theorem}
        Let G be a connected d-regular non-Eulerian graph. If $\bar{G}$ is connected, then $\bar{G}$ is Eulerian.
    \end{theorem}

    \begin{proof}
        We know that Eulerian is equivalent to being connected and having all vertices of even degree.
        Since $G$ is connected and $d$-regular, we only need to show that $d$ must be even.
        We also know that we cannot have a odd-regular graph on odd vertices, so $n$ must be even.
        Thus $\bar{G}$ has $n - 1 - d$ edges, therefore $d$ must be even as the number of vertices cannot be odd. 
        Thus, $\bar{G}$ is Eulerian.
    \end{proof}

    \begin{theorem}
        If $G - |S|$ has more than $|S|$ components, then $G$ is not hamiltonian.
    \end{theorem}

    \begin{proof}
        If for all non-adjacent vertices u,v in $G$, $\deg(u) + \deg(v) \ge n$.
        Then, $\delta(G) \ge \frac{n}{2} \implies G$ is hamiltonian.
    \end{proof}

    \begin{theorem}
        There exists a graph with $\kappa(G) \ge n$ but $G$ is not hamiltonian.
    \end{theorem}

    \begin{proof}
        Consider K$_{n,n+1}$, $\kappa(G) = n$ but $G$ is not hamiltonian.
    \end{proof}

    \begin{theorem}
        There exists a graph with $\kappa(G) \ge n$ but $G$ does not contain a hamiltonian path.
    \end{theorem}

    \begin{proof}
        Consider K$_{n,n+2}$ where $\kappa(G) = n$ but $G$ does not contain a hamiltonian path.
    \end{proof}

    \begin{theorem}
        Every tree has at most 1 perfect matching.
    \end{theorem}

    \begin{proof}
        Recall that a perfect matching is a matching that uses every vertex.
        Suppose for the sake of contradiction that there exists at least 2 perfect matchings.
        Then, there must exist a vertex that is matched to 2 different vertices in the 2 perfect matchings.
    \end{proof}

    \begin{theorem}
        The number of spanning trees is equal to 
        \[ \frac1n \lambda_1 \lambda_2 \dots \lambda_{n-1}\]
    \end{theorem}

    \begin{example}
        Consider the adjacency matrix below:
        \[\begin{bmatrix}
            2 & -1 & 0 & -1 \\
            -1 & 2 & -1 & 0 \\
            0 & -1 & 2 & -1 \\
            -1 & 0 & -1 & 2
        \end{bmatrix} \]
    \end{example}

    \begin{theorem}
        We know that $\kappa(G) \le \lambda(G) \le \delta(G) \le \lfloor{\frac{2m}{n}}\rfloor$
        If G is a tree, then all terms are equal.
    \end{theorem}

    \begin{proof}
        If G is a tree, then $\kappa(G) = \lambda(G) = \delta(G) = n - 1$.
    \end{proof}

\end{document}