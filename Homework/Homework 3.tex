\documentclass[boxes]{rutgers_hw}
\usepackage{rutgers}
% \usepackage[none]{hyphenat} % Use to avoid hyphens



\author{Rohan Karamel}
\netid{rak218} 
\assignment{Homework 3}
\date{\today}
\course{Graph Theory}
\semester{Spring 2024}
\instructor{Professor Benjamin Gunby}
\institution{Rutgers University}

\newtheorem*{lemma}{Lemma}
\newtheorem*{solutions}{Solution}

\begin{document}

  \maketitle

  \begin{exern}{1a}
    
  \end{exern}
  \begin{solutions}
    Here's an adjacency matrix, A, of G\@:
    \[
      \begin{bmatrix}
        0 & 1 & 0 & 0 & 1 \\
        1 & 0 & 1 & 0 & 0 \\
        0 & 1 & 0 & 1 & 0 \\
        0 & 0 & 1 & 0 & 1 \\
        1 & 0 & 0 & 1 & 0
      \end{bmatrix} 
    \]
  \end{solutions}

  \begin{exern}{1b}

  \end{exern}
  \begin{solutions}
    Let the first row and column of the matrix $A$ signify the vertex $a$.
    Likewise, let the third row and column signify the vertex $b${(such that the distance between $a$ and $b$ is 2)}. 
    To find the number of walks of length 30, we can just find the $(1,3)$ or $(3,1)$ element of the matrix $A^{30}$.
    Using a computer, the matrix $A^{30}$ is
    \[
      \begin{bmatrix}
        215492564 & 214146295 & 214978335 & 214978335 & 214146295 \\
        214146295 & 215492564 & 214146295 & 214978335 & 214978335\\
        214978335 & 214146295 & 215492564 & 214146295 & 214978335 \\
        214978335 & 214978335 & 214146295 & 215492564 & 214146295 \\
        214146295 & 214978335 & 214978335 & 214146295 & 215492564
      \end{bmatrix}
    \]

    Thus, the number of walks of length 30 from $a$ to $b$ is $214\,978\,335$.
  \end{solutions}

  \pagebreak

  \begin{exern}{2}
    
  \end{exern}
  \begin{proof}
    Let T be the transition probability matrix of G. To prove that $\frac{\deg(v)}{2m}$ is a steady state, we need to show that
    \[ 
      Tx = x, \hspace{5 mm} x = 
      \begin{bmatrix}
        \frac{\deg(v_1)}{2m} \\ \frac{\deg(v_2)}{2m} \\ \vdots \\ \frac{\deg(v_n)}{2m}
      \end{bmatrix}
    \]
    Now, we know that
    \[
      Tx = 
      \begin{bmatrix}
        \sum_{k=1}^n T_{1,i} \cdot \frac{\deg(v_1)}{2m} \\
        \sum_{k=1}^n T_{2,i} \cdot \frac{\deg(v_2)}{2m} \\
        \vdots \\
        \sum_{k=1}^n T_{n,i} \cdot \frac{\deg(v_n)}{2m}
      \end{bmatrix}
    \]
    By construction, we also know that the $(i,j)$-element of T is equal to $\frac{1}{\deg(v_i)}$ if $i \neq j$ and $v_i$ is connected to $v_j$.
    Let $f(u,v)$ signify that the vertices $u$ and $v$ are connected. 
    Therefore, we can simplify $Tx$ to 
    \[ 
      \begin{bmatrix}
        \sum_{k:f(v_1,v_k)} \frac{1}{\deg(v_1)} \cdot \frac{\deg(v_1)}{2m} \\
        \sum_{k:f(v_2,v_k)} \frac{1}{\deg(v_2)} \cdot \frac{\deg(v_2)}{2m} \\
        \vdots \\
        \sum_{k:f(v_n,v_k)} \frac{1}{\deg(v_n)} \cdot \frac{\deg(v_n)}{2m}
      \end{bmatrix}
    \]
    This can be written as
    \[ 
      \begin{bmatrix}
        \sum_{k:f(v_1,v_k)} \frac{1}{2m} \\
        \sum_{k:f(v_2,v_k)} \frac{1}{2m} \\
        \vdots \\
        \sum_{k:f(v_n,v_k)} \frac{1}{2m}
      \end{bmatrix}
    \]
    Because there are $\deg(v_i)$ terms of the $i^{th}$ sum, this equals
    \[
      \begin{bmatrix}
        \frac{\deg(v_1)}{2m} \\ \frac{\deg(v_2)}{2m} \\ \vdots \\ \frac{\deg(v_n)}{2m}
      \end{bmatrix}
    \]

    Thus, is equivalent to $x$ and we are done.
    
  \end{proof}

  \begin{exern}{3a}
    
  \end{exern}
  \begin{proof}
    We know that because G has 6 vertices and non-bipartite, it must contain an odd cycle. 
    Notably, a 3-cycle (triangle) or a 5-cycle because G has only 6 vertices. 
    Assume for the sake of contradiction that G doesn't contain a triangle. 
    We can form G by starting with $C_5$ as it must contain the 5-cycle.  
    However, to make G 3-regular, we must connect more edges. 
    But there is no possible edge you can add between vertices that will not create a 3-cycle.
    Thus, a contradiction forms. Therefore G must contain a triangle.
  \end{proof}

  \begin{exern}{3b}
    
  \end{exern}
  \begin{proof}
    3 edges because U needs at least 3 edges to make the triangle. U also has at most 3 edges connected to each of its vertices as a triangle is also the complete graph on 3 vertices. Therefore, there is exactly 3 edges. \\
    Because each vertex has degree 3, and the vertices in U connect to two other vertices in U, each of those vertices must connect to exactly 1 other vertex in W. Therefore there is 3 edges that connect a vertex in U and in W, 1 for each vertex in U. \\
    We have sorted that all the vertices in U are connected to two vertices in U and 1 in W. Now, let's say all of the vertices in U connected to the same vertex in W. 
    Then it would be impossible for the other vertices in W to be degree 3, as they can only connect to each other and are max degree 2.
    Now, suppose that 2 vertices in U connected to the same vertex in W.
    Then, the vertex in W with no connection to U could only connect to the other two vertices in W, which would be max degree 2.
    Therefore, we have shown that each vertex in U connects to a unique vertex in W. 
    Because of this, it follows that each vertex in W must connect to the other two vertices in W to be degree 3. \\
    We know W forms a triangle because each of the three vertices in W is connected to two other vertices in W. 
    This creates a 3-cycle/triangle.
  \end{proof}

  \pagebreak

  \begin{exern}{3c}
    
  \end{exern}
  \begin{proof}
    Let the smaller triangle in the drawing be U. Let the bigger triangle be W. 
    We can see that the drawing follows all the same rules as G does.
    This includes, 3-regular, non-bipartite, and all the rules we showed in 3b.
    We can see the 3 edges that connect vertices in U which form a triangle.
    We can see the 3 edges that connect a vertex in U with a unique vertex in W (and vice-versa).
    We can see the 3 edges that connect vertices in W which form a triangle.
  \end{proof}


\end{document}
